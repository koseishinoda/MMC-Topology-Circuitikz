\documentclass{standalone}
\usepackage[siunitx, RPvoltages, american, cute inductors, straight voltages]{circuitikz}
\usepackage{xparse}
\usepackage{tikz}
\usepackage{ifthen}

\makeatletter
\ctikzset{current arrow color/.initial=black}% create key
\tikzset{every text node part/.style={align=center}}

%Ref
%dahsed line
%https://tex.stackexchange.com/questions/45275/tikz-get-values-for-predefined-dash-patterns


\begin{document}
 
\begin{circuitikz}

\NewDocumentCommand{\varr}{O{blue} D(){below} m m}{% color, where, name, label
    \draw [thin, #1, -Triangle,] (#3-Vfrom) .. controls (#3-Vcont1) and (#3-Vcont2).. (#3-Vto) node [midway, sloped, #2] {#4};
    }

% %Figure dimensions
% \def\figHt{25}
% \def\figWd{25}
% %Comment next three lines once typesetting finished
% \draw[step=1,very thin, black!20] (-0.5,-0.5) grid (\figWd,\figHt);
% \foreach \x in {0,...,\figWd} {\node [anchor=north] at (\x,-0.5) {\x};}
% \foreach \y in {0,...,\figHt} {\node [anchor=east] at (-0.5,\y) {\y};}

%%%%%%%%%%%%%%%%%%%%%%%%%%%%%%%%%%%%%%%%%%%%%%%%%%%%%%%%%%%%%%%%%%%%%%%
\ctikzset{resistors/scale=0.8}

\tikzset{
    sm/.style={
        draw,
        rectangle,
        inner sep=0,
        minimum width=1.1cm,
        %maximum width=1.3cm,
        minimum height=0.5cm,
    }
}
%%%%%%%%%%%%%%%%%%%%%%%%%%%%%%%%%%%%%%%%%%%%%%%%%%%%%%%%%%%%

\newcommand\mySM[3]{
    % Create a node with the given style and label
    \node[sm] (#1) at ($#2+(0.3+0.55, 0)$) {\small #3};
    % Draw the connections to the left
    \draw ($(#1.west)+(0,+0.1)$) -- ++(-0.3,0) coordinate(#1-in);
    \draw ($(#1.west)+(0,-0.1)$) -- ++(-0.3,0) coordinate(#1-out);
}
\newcommand\myPtoGV[2]{
    \node (start) at (#1) {};
    \node[trarrow, rotate=+90, anchor=tip] at (start) {};
    \draw (start) -- ++(0,-0.5) to [short, l=#2] ++(0,-0.1) node[ground]{}; 
}

\newcommand\myPtoGVinv[2]{
    \node (start) at (#1) {};
    \node[trarrow, rotate=-90, anchor=tip] at (start) {};
    \draw (start) -- ++(0,+0.5) to [short, l_=#2] ++(0,+0.1) node[ground, rotate=180]{}; 
}

\newcommand\myPtoNV[3]{
    \node (start) at (#1) {};
    \node[trarrow, rotate=+90, anchor=tip] at (start) {};
    \draw (start) -- ($(#2)+(0,+0.5)$) to [short, -o, l=#3] (#2); 
}
%%%%%%%%%%%%%%%%%%%%%%%%%%%%%%%%%%%%%%%%%%%%%%%%%%%%%%%%%%%%

\ctikzset{resistors/scale=0.64}
\ctikzset{resistors/thickness=1}
\ctikzset{inductors/scale=0.8}
\ctikzset{inductors/thickness=1}

%\def\legtop{12}
%\def\legbot{0}

%Leg phase a 
\newcommand\nAdjust{0}


\NewDocumentCommand{\drawPhase}{m m} {% 1: x position, 2: phase label


    \ifthenelse{\equal{#2}{a}}{\renewcommand\nAdjust{+0.6}}{} % Adjust for phase a
    \ifthenelse{\equal{#2}{b}}{\renewcommand\nAdjust{+0.0}}{} % Adjust for phase b
    \ifthenelse{\equal{#2}{c}}{\renewcommand\nAdjust{-0.6}}{} % Adjust for phase c


    %Arm u

    \path (#1, 7+\nAdjust) coordinate (n#2);
    \draw (n#2) to [short, *-] ++ (0,+0.5-\nAdjust)
    to [L, l_=$L_{arm}$, mirror] ++(0,+1.2)
    to [R, l_=$R_{arm}$] ++(0,+0.9)
    to [short] ++(0, +0.2) coordinate (n#2SMu);

    \mySM{SM1#2u}{($(n#2SMu)+(0,+1.55)$)}{$SM_{1#2}^u$}
    \mySM{SM2#2u}{($(n#2SMu)+(0,+0.90)$)}{$SM_{2#2}^u$}
    \mySM{SMN#2u}{($(n#2SMu)+(0,+0.10)$)}{$SM_{N#2}^u$}

    \draw (SM1#2u-in) to [short, i<=$i_{#2}^{u}$, -*] ++(0, +0.9) coordinate (leg_#2_top);
    \draw (SM1#2u-out) to (SM2#2u-in);
    \draw [densely dashed] (SM2#2u-out) to (SMN#2u-in);

    \node (vm#2u_top) at ($(SM1#2u)+(-1.1,+0.4)$) {};
    \node (vm#2u_bot) at ($(SMN#2u)+(-1.1,-0.4)$) {};
    \draw(vm#2u_top) to [open, v=$u_{m#2}^u$] (vm#2u_bot);

    %Arm l

    \draw (n#2) to [short, *-] ++ (0,-0.5-\nAdjust)
    to [L, l=$L_{arm}$] ++(0,-1.2)
    to [R, l=$R_{arm}$] ++(0,-0.9)
    to [short] ++(0, -0.2) coordinate (n#2SMl);

    \mySM{SM1#2l}{($(n#2SMl)+(0,-0.10)$)}{$SM_{1#2}^l$}
    \mySM{SM2#2l}{($(n#2SMl)+(0,-0.75)$)}{$SM_{2#2}^l$}
    \mySM{SMN#2l}{($(n#2SMl)+(0,-1.55)$)}{$SM_{N#2}^l$}

    \draw (SM1#2l-out) to (SM2#2l-in);
    \draw [densely dashed] (SM2#2l-out) to (SMN#2l-in);
    \draw (SMN#2l-out) to [short, i_=$i_{#2}^{l}$, -*] ++(0,-0.9) coordinate (leg_#2_bot);

    \node (vm#2l_top) at ($(SM1#2l)+(-1.1,+0.3)$) {};
    \node (vm#2l_bot) at ($(SMN#2l)+(-1.1,-0.3)$) {};
    \draw(vm#2l_top) to [open, v=$u_{m#2}^l$] (vm#2l_bot);

}

\def\phasecenter{15}

% Draw the MMC legs
\drawPhase{\phasecenter-2.5}{a}
\drawPhase{\phasecenter}{b}
\drawPhase{\phasecenter+2.5}{c}

% AC transformer
% Set the transformer size
\ctikzset{bipoles/oosourcetrans/width=1.0824}  % default is 0.8
\ctikzset{bipoles/oosourcetrans/height=1.0824} % default is 0.8


% Draw the transformer
\def\xshifticonv{-1.2}
\draw (nb) to [short] ++(-2.5, 0) coordinate (nnb)
to [short, i=$i_{b}^{con}$] ++(\xshifticonv, 0) 
to [short] ++(-1.8, 0) to[oosourcetrans, name=TransAC, prim=wye, sec=delta, invert] ++(-2.0,0);

% Define the offset for the three-phase lines
\def\phaseOffset{0.1}

% AC voltage source to the left
\draw (TransAC.left) ++(-1.5,0.0) node[vsourcesinshape, rotate=90, scale=1.2] (Vac) {} ++(-1.0,0) coordinate (leftEnd);
\node[above] at (Vac.east) {$u^{ac}$};  % Label "Vac" above the AC voltage source

% Three-phase connection from the AC voltage source to the transformer
% Phase 1
\draw ([yshift=\phaseOffset cm]TransAC.left) to[short, -, i=$ $] ([yshift=\phaseOffset cm]Vac.south);
% Phase 2 (middle, no offset)
\draw (TransAC.left) to[short, -, i=$ $] (Vac.south);
% Phase 3
\draw ([yshift=-\phaseOffset cm]TransAC.left) to[short, -, i=$ $] ([yshift=-\phaseOffset cm]Vac.south);

\node at ($(Vac.south)!0.5!(TransAC.left)+(0,0.4)$) {$i^{ac}$};


\draw (na) 
to [short, i=$i_{a}^{con}$] ++(\xshifticonv, 0) 
to [short] ++(-1.8, 0) to [short]($(TransAC.centersec)!1!+60:(TransAC.right)$);

\draw (nc) to [short] ++(-5.0, 0) coordinate (nnc)
to [short, i=$i_{c}^{con}$] ++(\xshifticonv, 0) 
to [short] ++(-1.8, 0) to [short]($(TransAC.centersec)!1!-60:(TransAC.right)$); 

\draw ($(TransAC.centerprim)+(-0.15,0.0)$) to [short] ++(-135:0.5) node[ground]{};

\node (vacon_s)[trarrow, rotate=+90, anchor=tip] at ($( na)+(-2.8+0.0,+0.0)$){};
\draw (vacon_s) to [short, -, l=$v_{an}^{con}$] ++(0.0,-0.5) to [short, -*] ++(0.0,-1.5) coordinate (VNaCon); 

\node (vbcon_s)[trarrow, rotate=+90, anchor=tip] at ($(nnb)+(-2.8+0.5,+0.0)$){};
\draw (vbcon_s) to [short, -, l=$v_{bn}^{con}$] ++(0.0,-0.5) to [short, -*] ($(VNaCon)+(0.5,0.0)$) coordinate (VNbCon); 

\node (vccon_s)[trarrow, rotate=+90, anchor=tip] at ($(nnc)+(-2.8+1.0,+0.0)$){};
\draw (vccon_s) to [short, -, l=$v_{cn}^{con}$] ++(0.0,-0.5) to [short, -*] ($(VNbCon)+(0.5,0.0)$) coordinate (VNcCon); 

%draw horizontal line just below
\draw ($(VNaCon)+(0,0)$) to [short]($(VNcCon)+(0,0)$);

\myPtoGV{VNbCon}{$v_{n}^{con}$}

 

% DC side of the converter
% positive pole
\draw (leg_a_top) to [short] (leg_c_top);
\draw (leg_c_top) to [short, -] ++(0.5, 0)
to [L,l=$L^{dc}$] ++(1.0,0)
to [short] ++(0.5, 0)
to [short, -o, i=$i^{dc}$] ++(1.0, 0) 
 node [right] {} coordinate (PoC_DCp);
\node (v_p_dc) at ($(PoC_DCp)+(-1.0,+0.0)$) {};
\myPtoGV{v_p_dc}{$v_{p}^{dc}$}

% negative pole
\draw (leg_a_bot) to [short] (leg_c_bot);
\draw (leg_c_bot) to [short, -] ++(0.5, 0)
%to [L,l=$L_{n}^{dc}$] ++(2,0)
to [short] ++(0.5, 0)
to [short, -o] ++(2.0, 0)
 node [right] {} coordinate (PoC_DCn);
\node (v_n_dc) at ($(PoC_DCn)+(-1.0,+0.0)$) {};
\myPtoGVinv{v_n_dc}{$v_{n}^{dc}$}

%pole to pole voltage

\draw ($(PoC_DCp)+(0,1.5)$) to [open, v^=$u^{dc}$] ($(PoC_DCn)+(0,-1.5)$);

%%%%%%%%%%%%%%%%%%%%%%%%%%%%%%%%%%%%%%%%%%%%%%%%%%%%%%%%%%%%%%%%%%%%%%%

\end{circuitikz}
  
\end{document}